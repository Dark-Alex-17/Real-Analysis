\documentclass[12pt,letterpaper]{article}
\usepackage[utf8]{inputenc}
\usepackage[english]{babel}
\usepackage{amsthm}
\usepackage{amsmath}
\usepackage{amsfonts}
\usepackage{amssymb}
\usepackage{graphicx}
\usepackage{array}
\usepackage[left=2cm, right=2.5cm, top=2.5cm, bottom=2.5cm]{geometry}
\usepackage{enumitem}
\newcommand{\st}{\ \text{s.t.}\ }
\newcommand{\abs}[1]{\left\lvert #1 \right\rvert}
\newcommand{\R}{\mathbb{R}}
\newcommand{\N}{\mathbb{N}}
\newcommand{\Q}{\mathbb{Q}}
\newcommand{\C}{\mathbb{C}}
\newcommand{\Z}{\mathbb{Z}}
\DeclareMathOperator{\sign}{sgn}
\newtheoremstyle{case}{}{}{}{}{}{:}{ }{}
\theoremstyle{case}
\newtheorem{case}{Case}
\theoremstyle{definition}
\newtheorem{definition}{Definition}[section]
\newtheorem{theorem}{Theorem}[section]
\newtheorem*{theorem*}{Theorem}
\newtheorem{corollary}{Corollary}[section]
\newtheorem{lemma}[theorem]{Lemma}
\newtheorem*{remark}{Remark}
\setlist[enumerate]{font=\bfseries}
\renewcommand{\qedsymbol}{$\blacksquare$}
\author{Alexander J. Tusa}
\title{Real Analysis Homework 5}
\begin{document}
	\maketitle
	\begin{enumerate}
	\item For the following sequences, i) write out the first 5 terms, ii) Use the Monotone Sequence Property to show that the sequences converges.
	\begin{enumerate}
		\item \textbf{Section 3.3}
		\begin{enumerate}
			\item[2)] Let $x_1 > 1$ and $x_{n+1} := 2-1/x_n$ for $n \in \N$. Show that $(x_n)$ is bounded and monotone. Find the limit.
			\\\\ The first five terms of this sequence are $x_1 \geq 2,x_2 \geq \frac{3}{2}, x_3 \geq \frac{4}{5}, x_4 \geq \frac{5}{4}, x_5 \geq \frac{6}{5}, \dots$. This sequence appears to be decreasing.
			\\\\Recall the Monotone Sequence Property:
			\begin{theorem*}{Monotone Sequence Property}
				A monotone sequence of real numbers is convergent if and only if it is bounded. Further,
				\begin{enumerate}
					\item If $X=(x_n)$ is a bounded increasing sequence, then
					\[\lim (x_n) = \sup \{x_n:n \in \N\}\]
					
					\item If $Y=(y_n)$ is a bounded decreasing sequence, then
					\[\lim (y_n) = \inf \{y_n : n \in \N \}\]
				\end{enumerate}
			\end{theorem*}
		
			To show that this sequence converges, we must first find the possible limit points (fixed points) of this sequence. So,
			\begin{align*}
				x&=2-\frac{1}{x} \\
				x^2 &= 2x -1 \\
				x^2 - 2x + 1 &= 0 \\
				(x-1)^2 &= 0
			\end{align*}
			Thus, $x=1$ is a possible limit of this sequence.
			\\\\Now, we will prove that $(x_n)$ is bounded by $1$, and since we hypothesized that $(x_n)$ is decreasing, we say that $(x_n)$ is bounded below by 1. 
			\begin{proof}
				We want to show that the sequence $(x_n)$ is bounded below by 1; that is, we want to show that $1 \leq x_n,\ \forall\ n \in \N$. We prove it by method of mathematical induction.
				\\\\\textbf{Basis Step:} Let $n=1$. Then
				\begin{align*}
					x_n &\geq x_{n+1}, &\text{by the definition of decreasing,} \\
					x_1 &\geq x_{1+1} \\
					x_1 &\geq x_2
				\end{align*}
				Since $x_1>1 \Rightarrow \frac{1}{x_1} < 1$, we have
					\[x_2 = 2-\frac{1}{x_1} > 1\]
					\[\Rightarrow 1 < x_2 < 2.\]
				Since $x_1>1$ and because $1 < x_2 < 2$, we have that $x_1 \geq x_2$.
				\\\\\textbf{Inductive Step:} Assume $1 < x_n < 2,\ \forall\ n \in \N$.
				\\\\\textbf{Show:} Now we want to show that $x_n \leq x_{n+1}$.
				\\So,
					\[1 < x_n <2\]
					\[1 > \frac{1}{x_n} > \frac{1}{2}\]
					\[-1 < -\frac{1}{x_n} < -\frac{1}{2}\]
					\[1 < 2-\frac{1}{x_n} < 2-\frac{1}{2} < 2\]
					\[1 < x_{n+1} < 2\]
				Thus we have that $(x_n)$ is bounded between 1 and 2.
			\end{proof}
		
			Now we need to show that $(x_n)$ is monotone decreasing; that is, we must show that $x_1 \geq x_2 \geq \dots \geq x_n$.
			
			\begin{proof}
				We want to show that $x_1 \geq x_2 \geq \dots \geq x_n,\ \forall\ n \in \N$. We prove it by method of mathematical induction.
				\\\\\textbf{Basis Step:} Let $n=1$. Then since $x_1 >1$ is given, we have that $\frac{1}{x_1} < 1$. This yields $x_2=2-\frac{1}{x_1}>1$, as was determined for the boundedness proof, and thus we have that $1 < x_2 < 2$. This means that $1 > \frac{1}{x_2} > \frac{1}{2}$, and since $\frac{1}{2} \leq \frac{1}{x_n}$, we have $x_2 \geq x_1$.
				\\\\\textbf{Inductive Step:} Assume $x_n \geq x_{n+1}\ \forall\ n \in \N$.
				\\\\\textbf{Show:} We now want to show that $x_{n+2} \leq x_{n+1}$.
				\\So,
				\[x_{n+2}=2-\frac{1}{x_{x+1}}\]
				Recall the inductive hypothesis, in that $x_n \geq x_{n+1} \Rightarrow \frac{1}{x_n} \leq \frac{1}{x_{n+1}}$. Thus,
				\[-\frac{1}{x_n} \geq -\frac{1}{x_{n+1}}\]
				\[\Rightarrow 2-\frac{1}{x_n} \leq 2 -\frac{1}{x_{n+1}}\]
				\[x_{n+1} \leq x_{n+2}\]
				$\therefore$ we have that $x_1 \geq x_2 \geq \dots \geq x_n,\ \forall\ n \in \N$.
			\end{proof}
			Thus $(x_n)$ is monotone decreasing.
			\\\\By the \textit{Monotone Sequence Property}, since we have shown that $(x_n)$ is both bounded (and thus converges), and that $(x_n)$ is monotone decreasing, we have that
			\begin{align*}
				\lim (x_n) &= \inf \{x_n: n \in \N\} \\
				&=\inf (1,2) \\
				&= 1
			\end{align*}
			Hence the sequence converges to the previously found possible limit of 1. \\
			
			\item[3)] Let $x_1 > 1$ and $x_{n+1} := 1 + \sqrt{x_n - 1}$ for $n \in \N$. Show that $(x_n)$ is decreasing and bounded below by $2$. Find the limit.
			\\\\The first 5 terms of this sequence are $x_1 \geq 2, x_2 \geq 2, x_3 \geq 2, x_4 \geq 2, x_5 \geq 2, \dots$. Notice the following, however:
			\begin{align*}
				x_{n+1} \leq x_n &\iff 1+\sqrt{x_n - 1} \leq x_n \\
				&\iff \sqrt{x_n -1} \leq x_n -1
			\end{align*}
			which we know is always true since the square root function is a decreasing function.
			\\\\Now we must find the possible limit points (fixed points) of this sequence. So,
			\begin{align*}
				x &= 1 + \sqrt{x-1} \\
				x-1 &= \sqrt{x-1} \\
				x-1 &= (x-1)^2 \\
				x-1 &= x^2 -2x +1 \\
				(x-1)-(x^2-2x+1) &= 0 \\
				-x^2+3x-2 &=0 \\
				-(x^2-3x+2) &= 0 \\
				-(x-1)(x-2) &= 0 \\
				(x-1)(x-2) &= 0
			\end{align*}
			Thus $x=1$, or $x=2$. These are the possible limits of $(x_n)$. Since we hypothesized that $(x_n)$ is decreasing, then we say that $(x_n)$ is bounded below by 2, since we are given that $x_1 > 1$.
			\\\\Now we will prove that $(x_n)$ is bounded below by 2.\\
			\begin{proof}
				We want to show that $(x_n)$ is bounded below by 1; that is, we want to show that $1 \leq x_n,\ \forall\ n \in \N$. We prove it by method of mathematical induction.
				\\\\\textbf{Basis Step:} Let $n=1$. Then we are given that $x_1 \geq 2$.
				\\\\\textbf{Inductive Step:} Assume that $x_n \geq 2,\ \forall\ n \in \N$.
				\\\\\textbf{Show:} We now want to show that $x_{n+1} \geq 2,\ \forall\ n \in \N$.
				\\\\So,
				\begin{align*}
					x_{n+1} &= 1+\sqrt{x_n -1} \\
					&\geq 1+\sqrt{2 -1} \\
					&=1 + 1 \\
					&= 2
				\end{align*}
				Thus, $x_n \geq 2,\ \forall\ n \in \N$. By the definition of boundedness, we have that $(x_n)$ is bounded below by 2.
			\end{proof}
			
			Since we have also shown earlier that $(x_n)$ is monotone decreasing, we have that by the monotone sequence property, since $(x_n)$ is bounded, $(x_n)$ converges, and since $(x_n)$ is monotone decreasing, we have:
			\begin{align*}
				\lim (x_n) &= \inf \{x_n:n \in \N\} \\
				&=2
			\end{align*}
			
			\item[7)] Let $x_1 := a>0$ and $x_{n+1} := x_n+1/x_n$ for $n \in \N$. Determine whether $(x_n)$ converges or diverges.
			\\\\The first 5 terms of this sequence are $x_1 \geq 1, x_2 \geq 2, x_3 \geq \frac{5}{2}, x_4 \geq \frac{29}{10}, x_5 \geq \frac{941}{290}, \dots$. This sequence appears to be increasing. We show this to be true as follows:
			\begin{align*}
				x_{n+1} \geq x_n &\iff x_n + \frac{1}{x_n} \geq x_n \\
				&\iff x_n^2 + 1 \geq x_n^2 \\
				&\iff 1 \geq 0
			\end{align*}
			which is true.
			However, notice that one of the terms of the sequence is $x_n$. We know that $x_n$ is an unbounded sequence. Thus, we can infer that $(x_n)$ is unbounded above. We show this as follows:
			\begin{align*}
				x_{n+1}^2 &= \left(x_n + \frac{1}{x_n} \right)^2 \\
				&= x_n^2+2+\frac{1}{x_n^2} \\
				&> x_n^2 +2
			\end{align*}
			Since:
			\[x_{n+1}^2 > x_n^2+2 > x_{n-1}^2 +4 > \dots > x_1^2+2 \cdot n = a^2+2 \cdot n\]
			\[\Downarrow\]
			\[x_n > \sqrt{a^2 + 2 \cdot (n-1)}\]
			Since the right hand side of this inequality is unbounded, the left hand side is also unbounded.
			\\\\Thus we have that this sequence $(x_n)$ is unbounded above.
			\\\\Since this sequence is increasing and unbounded above, we have that the sequence is divergent.\\
			
			\item[8)] Let $(a_n)$ be an increasing sequence, $(b_n)$ be a decreasing sequence, and assume that $a_n \leq b_n$ for all $n \in \N$. Show that $\lim (a_n) \leq \lim (b_n)$, and thereby deduce the Nested Intervals Property 2.5.2 from the Monotone Convergence Theorem 3.3.2.
			\\\\Since $(a_n)$ is an increasing sequence, we know that $(a_1 \leq a_2 \leq \dots \leq a_n)$, and since $(b_n)$ is a decreasing sequence, we know that $(b_1 \geq b_2 \geq \dots \geq b_n)$. Also, since we have that $a_n \leq b_n,\ \forall\ n \in \N$, we know that $(a_n)$ is bounded above by $(b_1)$. Thus, by the \textit{Monotone Convergence Theorem}, we know that
			\[\lim (a_n) = \sup \{a_n: n \in \N\}\]
			Also, since $(b_n)$ is a decreasing sequence such that it is bounded below by $(a_1)$, by the \textit{Monotone Convergence Theorem}, we have
			\[\lim (b_n) = \inf \{b_n: n \in \N\}\]
			Recall Theorem 3.2.5:
			\begin{theorem*}
				If $X=(x_n)$ and $Y=(y_n)$ are convergent sequences of real numbers and if $x_n \leq y_n\ \forall\ n \in \N$, then $\lim (x_n) \leq \lim (y_n)$.
			\end{theorem*}
			Also, recall the \textit{Nested Intervals Property}:
			\begin{theorem*}
				If $I_n=[a_n,b_n],\ n \in \N$, is a nested sequence of closed bounded intervals, then there exists a number $\xi \in \R \st \xi \in I_n\ \forall\ n \in \N$.
			\end{theorem*}
			Note that we have a nested sequence of closed, bounded intervals: $[a_n, b_n],\ n \in \N$. Since we showed that $\lim (a_n) \leq \lim (b_n)$, (and we are given that $(a_n)$ is increasing and $(b_n)$ is decreasing), we know that there exists $\xi$ such that
			\[\lim (a_n) \leq \xi \leq \lim (b_n)\]
			which means that $\xi \in [a_n, b_n],\ \forall\ n \in \N$.
		\end{enumerate}
		
		\item $a_1 = 1,\ a_{n+1}=\frac{a_n^2+5}{2a_n}$
		\\\\The first 5 terms of this sequence are $1, 3, \frac{7}{3}, \frac{47}{21}, \frac{2207}{987}, \dots$. This is a decreasing sequence.
		\\\\First, we must find the possible limits (fixed points) of the sequence. So,
		\begin{align*}
			a&=\frac{a^2+5}{2a} \\
			2a^2 &= a^2+5 \\
			a^2 &= 5 \\
			a &= \pm \sqrt{5}
		\end{align*}
		Since we're given that $a_1=1$, we know that the most likely lower bound will be $\sqrt{5}$.
		\\\\Now we want to show that $(a_n)$ is bounded below by $\sqrt{5}$.
		\begin{proof}
			We want to show that $a_n \geq \sqrt{5},\ \forall\ n \in \N$. We prove it by method of mathematical induction.
			\\\\\textbf{Basis Step:} Since $1 \geq \sqrt{5}$, we have that $a_1 \geq \sqrt{5}$
			\\\\\textbf{Inductive Step:} Assume that $a_n \geq \sqrt{5}\ \forall\ n \in \N$.
			\\\\\textbf{Show:} We want to show that $a_{n+1} \geq \sqrt{5}\ \forall\ n \in \N$. So, 
			\[a_{n+1} = \frac{a_n^2 + 5}{2a_n}\]
			\begin{align*}
				(a_n-\sqrt{5})^2 &\geq 0 \\
				a_n^2 -2\sqrt{5}a_n +5 &\geq 0 \\
				a_n^2 +5 &\geq 2\sqrt{5}a_n \\
				\Downarrow \\
				\frac{a_n^2+5}{2a_n} &\geq \frac{2\sqrt{5}a_n}{2a_n} \\
				\frac{a_n^2+5}{2a_n} &\geq \sqrt{5} \\
				a_{n+1} \geq \sqrt{5}
			\end{align*}
			Thus we have that $(a_n)$ is bounded below by $\sqrt{5}$.
		\end{proof}
		Now we must show that $(a_n)$ is monotone decreasing.\\
		\begin{proof}
			We want to show that $(a_n)$ is monotone decreasing; that is, we want to show that $(a_2 \geq a_3 \geq \dots \geq a_n),\ \forall\ n \geq 2$. We prove it by method of mathematical induction.
			\\\\\textbf{Basis Step:} Since $3 \geq \frac{7}{3}$, we have that $a_2 \geq a_3$.
			\\\\\textbf{Inductive Step:} Assume that $a_n \geq a_{n+1},\ \forall\ n \geq 2$.
			\\\\\textbf{Show:} We want to show that $a_{n+2} \leq a_{n+1},\ \forall\ n \geq 2$.
			\\So,
			\[a_{n+2} = \frac{a_{n+1}^2+5}{2a_{n+1}} \leq \frac{a_n^2+5}{2a_n}\]
			Since we have:
			\begin{align*}
				a_{n+1} &\geq \sqrt{5}, &\text{by the previous proof of boundedness} \\
				a_{n+1}^2 &\geq 5
			\end{align*}
			We can equivalently write the inequality as
			\[\frac{a_{n+1}^2+5}{2a_{n+1}} \leq \frac{a_{n+1}^2+a_{n+1}^2}{2a_{n+1}}=a_{n+1}\]
			Thus we have that $(a_n)$ is monotone decreasing.
		\end{proof}
		Since $(a_n)$ is both monotone decreasing and bounded, we have
		\begin{align*}
			\lim (a_n) &= \inf \{a_n:n \in \N\} \\
			&= \sqrt{5}
		\end{align*} 
		
		\item $a_1 = 5,\ a_{n+1}=\sqrt{4+a_n}$
		\\\\The first 5 terms of this sequence are $5, \sqrt{5}, \sqrt{7}, \frac{\sqrt{57}}{3}, \frac{\sqrt{2751}}{21}, \dots$. This sequence is decreasing.
		\\\\First, we must find the possible limits (fixed points) of the sequence. So,
		\begin{align*}
			a&=\sqrt{4+a} \\
			a^2 &= 4 + a \\
			a^2 -a - 4 &= 0 \\
		\end{align*}
		So we have that $a=\frac{1}{2} - \sqrt{17}$, or $a=\frac{1}{2}+\sqrt{17}$. Since we're given that $a_1=5$, we infer that $\frac{1}{2}+\sqrt{17}$ is a possible limit (fixed point) of $(a_n)$.
		\\\\Now we want to show that $(a_n)$ is bounded below by $\frac{1}{2}+\sqrt{17}$.
		\begin{proof}
			We want to show that $a_n \geq \frac{1}{2}+\sqrt{17},\ \forall\ n \in \N$. We prove it by method of mathematical induction.
			\\\\\textbf{Basis Step:} $5 \geq \frac{1}{2}+\sqrt{17}$ yields that $a_1 \ge \frac{1}{2}+\sqrt{17}$
			\\\\\textbf{Inductive Step:} Assume $a_n \geq \frac{1}{2}+\sqrt{17},\ \forall\ n \in \N$.
			\\\\\textbf{Show:} We want to show that $a_{n+1} \geq \frac{1}{2}+\sqrt{17},\ \forall\ n \in \N$.
			\\So,
			\begin{align*}
				a_{n+1} \ge \sqrt{4+a_n}
			\end{align*}
			
		\end{proof}
	\end{enumerate}
	
	\item 
	\begin{enumerate}
		\item Show $a_n=\frac{3 \cdot 5 \cdot 7 \cdot \dots (2n-1)}{2 \cdot 4 \cdot 6 \dots (2n)}$ converges to $A$ where $0 \leq A < 1/2$.
		
		\item Show $b_n = \frac{2 \cdot 4 \cdot 6 \dots (2n)}{3 \cdot 5 \cdot 7 \dots (2n+1)}$ converges to $B$ where $0 \leq B < 2/3$.
	\end{enumerate}

	\item \textbf{Section 3.4}
	\begin{enumerate}
		\item[1)] Give an example of an unbounded sequence that has a convergent subsequence.
		
		\item[3)] Let $(f_n)$ be the Fibonacci sequence of Example 3.1.2(d), and let $x_n := f_{n+1}/f_n$. Given that $\lim (x_n) =L$ exists, determine the value of $L$.
		
		\item[4a)] Show that the sequence $(1-(-1)^n+1/n)$ converges.
		
		\item[16)] Give an example to show that Theorem 3.4.9 fails if the hypothesis that $X$ is a bounded sequences is dropped.
		
		\item[18)] Show that if $(x_n)$ is a bounded sequence, then $(x_n)$ converges if and only if $\lim \sup (x_n) = \lim \inf (x_n)$.
		
		\item[19)] Show that if $(x_n)$ and $(y_n)$ are bounded sequences, then
		\[\lim \sup (x_n + y_n) \leq \lim \sup (x_n) + \lim \sup (y_n).\]
		Give an example in which the two sides are not equal.
	\end{enumerate}
	\item 
	\begin{enumerate}
		\item Show that $x_n=e^{\sin (5n)}$ has a convergent subsequence.
		
		\item Give an example of a bounded sequence with three subsequences converging to three different numbers.
		
		\item Give an example of a sequence $x_n$ with $\lim \sup x_n = 5$ and $\lim \sup x_n = -3$.
		
		\item Let $\lim \sup x_n = 2$. True or False: if $n$ is sufficiently large, then $x_n > 1.99$.
		
		\item Compute the infimum, supremum, limit infimum, and limit supremum for $a_n = 3 - (-1)^n - (-1)^n/n$. 
	\end{enumerate}

	\item 
	\begin{enumerate}
		\item If $a_n$ and $b_n$ are strictly increasing, then $a_n + b_n$ is strictly increasing.
		
		\item If $a_n$ and $b_n$ are strictly increasing, then $a_n \cdot b_n$ is strictly increasing.
		
		\item If $a_n$ and $b_n$ are monotonic, then $a_n + b_n$ is monotonic.
		
		\item If $a_n$ and $b_n$ are monotonic, then $a_n \cdot b_n$ is monotonic.
		
		\item If a monotone sequence is bounded, then it is convergent.
		
		\item If a bounded sequence is monotone, then it is convergent.
		
		\item If a convergent sequence is monotone, then it is bounded.
		
		\item If a convergent sequence is bounded, then it is monotone.
	\end{enumerate}
	\end{enumerate}
\end{document}