\documentclass[12pt,letterpaper]{article}
\usepackage[utf8]{inputenc}
\usepackage{pgfplots}
\usepackage[english]{babel}
\usepackage{amsthm}
\usepackage{cancel}
\usepackage{mathtools}
\usepackage{amsmath}
\usepackage{amsfonts}
\usepackage{amssymb}
\usepackage{graphicx}
\usepackage{array}
\usepackage[left=2cm, right=2.5cm, top=2.5cm, bottom=2.5cm]{geometry}
\usepackage{enumitem}
\usepackage{mathrsfs}
\newcommand{\limx}[2]{\displaystyle\lim\limits_{#1 \to #2}}
\newcommand{\st}{\ \text{s.t.}\ }
\newcommand{\abs}[1]{\left\lvert #1 \right\rvert}
\newcommand{\R}{\mathbb{R}}
\newcommand{\N}{\mathbb{N}}
\newcommand{\Q}{\mathbb{Q}}
\newcommand{\C}{\mathbb{C}}
\newcommand{\Z}{\mathbb{Z}}
\newcommand{\dotp}{\dot{\mathcal{P}}}
\newcommand{\dotq}{\dot{\mathcal{Q}}}
\newcommand{\dist}{\text{dist}}
\DeclareMathOperator{\sign}{sgn}
\newtheoremstyle{case}{}{}{}{}{}{:}{ }{}
\theoremstyle{case}
\newtheorem{case}{Case}
\newtheorem{case*}{Case}
\theoremstyle{definition}
\newtheorem{definition}{Definition}[section]
\newtheorem{theorem}{Theorem}[section]
\newtheorem*{theorem*}{Theorem}
\newtheorem{corollary}{Corollary}[section]
\newtheorem*{corollary*}{Corollary}
\newtheorem{lemma}[theorem]{Lemma}
\newtheorem*{lemma*}{Lemma}
\newtheorem*{remark}{Remark}
\setlist[enumerate]{font=\bfseries}
\renewcommand{\qedsymbol}{$\blacksquare$}
\author{Alexander J. Tusa}
\title{Real Analysis II Homework 5}
\begin{document}
	\maketitle
	\begin{enumerate}
		\item Find the sum of the following series.
		\begin{enumerate}
			\item (pr. 3a) $\displaystyle\sum_{n=1}^{\infty} \frac{1}{n^2+3n+2}$
			\begin{align*}
				\sum_{n=1}^{\infty} \frac{1}{n^2+3n+2} &= \sum_{n=1}^{\infty} \frac{1}{(n+1)(n+2)} \\
				&\Downarrow \\
				\frac{1}{n^2+3n+2}&=\frac{A}{n+1} + \frac{B}{n+2} \\
				1&=A(n+2)+ B(n+2) \\
				1&=An+2A+Bn+2B \\
				1&=An+Bn+2A+2B \\
				1&= (A+B)n+(A+B)2 \\
			\end{align*}
			\[\begin{cases}
			0=A+B \\
			1=A+B
			\end{cases} \implies
			\begin{cases}
			A=1 \\
			B=-1
			\end{cases}\]
			\begin{align*}
				&=\sum_{n=1}^{\infty} \frac{1}{n+1}-\frac{1}{n+2} \\
				&= \left(\frac{1}{2}-\cancel{\frac{1}{3}}\right) + \left(\cancel{\frac{1}{3}}-\cancel{\frac{1}{4}}\right) + \dots + \left(\cancel{\frac{1}{n}}-\cancel{\frac{1}{n+1}}\right)+ \left(\cancel{\frac{1}{n+1}}-\frac{1}{n+2}\right) \\
				&= \limx{n}{\infty} \frac{1}{2}+\frac{1}{n+2} \\
				&= \frac{1}{2}
			\end{align*}
			\item $\displaystyle\sum_{n=1}^{\infty} \frac{1}{(3n-2)(3n+1)}$
			\begin{align*}
				\frac{1}{(3n-2)(3n+1)} &= \frac{A}{3n-2} + \frac{B}{3n+1} \\
				1&= A(3n+1)+B(3n-2) \\
				1&= 3An+3Bn+A-2B
			\end{align*}
			\[\begin{cases}
			3A+3B=0 \\
			1A-2B=1
			\end{cases} \implies \begin{cases}
			A=\frac{1}{3} \\
			B=\frac{-1}{3}
			\end{cases}\]
			\begin{align*}
				\frac{1}{(3n-2)(3n+1)} &= \frac{1}{9n-6} - \frac{1}{9n+3} \\
				\sum_{n=1}^{\infty} \frac{1}{(3n-2)(3n+1)} &= \sum_{n=1}^{\infty} \frac{1}{9n-6} - \frac{1}{9n+3} \\
				&= \left(\frac{1}{3}-\cancel{\frac{1}{12}}\right) + \left(\cancel{\frac{1}{12}}-\cancel{\frac{1}{21}}\right) + \dots\\
				&+ \left(\cancel{\frac{1}{9n-15}}-\cancel{\frac{1}{9n-6}}\right) + \left(\cancel{\frac{1}{9n-6}}-\frac{1}{9n+3}\right) \\
				&= \limx{n}{\infty} \frac{1}{3}-\frac{1}{9n+3} \\
				&= \frac{1}{3}
			\end{align*}
			\item $\displaystyle\sum_{n=1}^{\infty} \frac{(-1)^{n+1}}{e^{n-1}}$
			\begin{align*}
				\sum_{n=1}^{\infty} \frac{(-1)^{n+1}}{e^{n-1}} &= \sum_{n=1}^{\infty} \frac{(-1)^n \cdot (-1)^1}{e^n\cdot e^{-1}} \\
				&=- \sum_{n=1}^{\infty} \frac{(-1)^n \cdot e}{e^n} \\
				&= - \sum_{n=1}^{\infty} (-1)^n \cdot \frac{e}{e^n} \\
				&= - \sum_{n=1}^{\infty} (-1)^n \cdot e^{1-n} \\
				&= \frac{e}{1+e}
			\end{align*}
			\item $\displaystyle\sum_{n=2}^{\infty} \frac{4^{n+1}}{9^{n-1}}$
			\begin{align*}
				\sum_{n=2}^{\infty} \frac{4^{n+1}}{9^{n-1}} &= \sum_{n=2}^{\infty} \frac{4^n \cdot 4^1}{9^n\cdot 9^{-1}} \\
				&= \sum_{n=2}^{\infty} \left(\frac{4}{9}\right)^n \cdot 4^1 \cdot 9^1 \\
				&= \sum_{n=2}^{\infty} \left(\frac{4}{9}\right)^n \cdot 36 \\
				&= \sum_{n=2}^{\infty} \left(\frac{4}{9}\right)^n \cdot 36 - 16-36 \\
				&= a\left(\frac{1}{1-r}\right) \\
				&= 36 \cdot \left(\frac{1}{1-\left(\frac{4}{9}\right)}\right) -52 \\
				&= \frac{36 \cdot 9}{5} - 52 \\
				&= \frac{324}{5} - 52 \\
				&= \frac{324}{5} - \frac{260}{5} \\
				&= \frac{64}{5}
			\end{align*}
			\item $\displaystyle\sum_{n=0}^{\infty} \frac{5^{n+1}+(-3)^n}{7^{n+2}}$
			\begin{align*}
				\sum_{n=0}^{\infty} \frac{5^{n+1}+(-3)^n}{7^{n+2}} &= \sum_{n=0}^{\infty} \frac{5^{n+1}}{7^{n+2}}+\frac{(-3)^n}{7^{n+2}} \\
				&= \sum_{n=0}^{\infty} \frac{5}{49} \cdot \left(\frac{5}{7}\right)^n + \frac{1}{49} \cdot \left(\frac{(-3)}{7}\right)^n \\
				&= \sum_{n=0}^{\infty} \frac{5}{49} \cdot \left(\frac{5}{7}\right)^n + \sum_{n=0}^{\infty} \frac{1}{49} \cdot \left(\frac{(-3)}{7}\right)^n \\
				&= \frac{5}{49} \cdot \frac{1}{1-\frac{5}{7}} + \frac{1}{49} \cdot \frac{1}{1-\frac{-3}{7}} \\
				&= \frac{5}{14} + \frac{1}{70} \\
				&= \frac{13}{35}
			\end{align*}
			\item $\displaystyle\sum_{n=2}^{\infty} \ln \frac{n^2-1}{n^2}$
			\begin{align*}
				\sum_{n=2}^{\infty} \ln \left(\frac{n^2-1}{n^2}\right) &= \sum_{n=2}^{\infty} \ln \left(\frac{(n-1)(n+1)}{n^2}\right) \\
				&= \sum_{n=2}^{\infty} \ln \left(\frac{\frac{n-1}{n}}{\frac{n}{n+1}}\right) \\
				&= \sum_{n=2}^{\infty} \ln \left(\frac{n-1}{n}\right) - \ln\left(\frac{n}{n+1}\right) \\
				&= \left(\ln \frac{1}{2} - \cancel{\ln \frac{2}{3}}\right) + \left(\cancel{\ln \frac{2}{3}} - \cancel{\ln \frac{3}{4}}\right) +\\
				&\dots + \left(\cancel{\ln \frac{n-2}{n-1}}-\cancel{\ln \frac{n-1}{n}}\right) + \left(\cancel{\ln \frac{n-1}{n}}-\ln \frac{n}{n+1}\right) \\
				&= \limx{n}{\infty} \ln \left(\frac{1}{2}\right) - \ln \left(\frac{n}{n+1}\right) \\
				&= \ln \left(\frac{1}{2}\right) - \limx{n}{\infty} \frac{n+1}{n} \cdot \frac{n+1-n}{(n+1)^2}, &\text{by L'Hospital's Rule} \\
				&= \ln \left(\frac{1}{2}\right) - \limx{n}{\infty} \frac{1}{n(n+1)} \\
				&= \ln\left(\frac{1}{2}\right) - 0 \\
				&= \ln\left(\frac{1}{2}\right) \\
				&\approx -0.693147
			\end{align*}
			\item $\displaystyle\sum_{n=2}^{\infty} \ln \frac{n(n+2)}{(n+1)^2}$
			\begin{align*}
				\sum_{n=2}^{\infty} \ln\left(\frac{n(n+2)}{(n+1)^2}\right) &= \sum_{n=2}^{\infty} \ln \left(\frac{\frac{n}{n+1}}{\frac{n+1}{n+2}}\right) \\
				&= \sum_{n=2}^{\infty} \ln\left(\frac{n}{n+1}\right) - \ln\left(\frac{n+1}{n+2}\right)  \\
				&= \left(\ln \frac{2}{3}-\cancel{\ln \frac{3}{4}}\right) + \left(\cancel{\ln \frac{3}{4}}-\cancel{\ln \frac{4}{5}}\right) + \dots \\
				&+ \left(\cancel{\ln \frac{n-1}{n}}-\cancel{\ln\frac{n}{n+1}}\right) + \left(\cancel{\ln \frac{n}{n+1}}-\ln\frac{n+1}{n+2}\right) \\
				&= \limx{n}{\infty}\ln \left(\frac{2}{3}\right) - \ln \left(\frac{n+1}{n+2}\right) \\
				&= \ln \left(\frac{2}{3}\right)-\limx{n}{\infty} \frac{n+2}{n+1} \cdot \frac{1\cdot (n+2) - (n+1)\cdot 1}{(n+2)^2}, &\text{by L'Hospital's Rule} \\
				&= \ln \left(\frac{2}{3}\right)- \limx{n}{\infty} \frac{1}{(n+1)(n+2)} \\
				&= \ln \left(\frac{2}{3}\right) - 0 \\
				&= \ln \left(\frac{2}{3}\right) \\
				&\approx -0.405465
			\end{align*}
			\item (pr. 3c) $\displaystyle\sum_{n=1}^{\infty} \frac{1}{n(n+1)(n+2)}$
			\begin{align*}
				\frac{1}{n(n+1)(n+2)} &= \frac{A}{n} + \frac{B}{n+1} + \frac{C}{n+2} \\
				1&=A(n+1)(n+2) + Bn(n+2)+Cn(n+1) \\
				1&=An^2+3An+2A+Bn^2+2Bn+Cn^2+Cn \\
				1&= An^2+Bn^2+Cn^2+3An+2Bn+Cn+2A
			\end{align*}
			\[\begin{cases}
			An^2+Bn^2+Cn^2=0 \\
			3An+2Bn+Cn=0 \\
			2A=1
			\end{cases} = \begin{cases}
			A=\frac{1}{2} \\
			B=-1 \\
			C=\frac{1}{2}
			\end{cases}\]
			\begin{align*}
				\sum_{n=1}^{\infty} \frac{1}{n(n+1)(n+2)} &= \sum_{n=1}^{\infty} \frac{1}{2n} -\frac{1}{n+1} + \frac{1}{2n+4} \\
				&= \left(\frac{1}{2}-\frac{1}{2}+\frac{1}{6}\right) + \left(\frac{1}{4}-\frac{1}{3}+\frac{1}{8}\right) + \left(\frac{1}{6}-\frac{1}{4}+\frac{1}{10}\right) \\
				&+ \dots + \left(\frac{1}{2n-2}+\frac{1}{2n+2}-\frac{1}{n}\right) + \left(\frac{1}{2n}+\frac{1}{2n+4}-\frac{1}{n+1}\right) \\
				&= \limx{n}{\infty} \frac{1}{4} + \frac{1}{2(n+1)(n+2)} \\
				&= \frac{1}{4} + 0 \\
				&= \frac{1}{4}
			\end{align*}
			\item $\displaystyle\frac{1}{1 \cdot 3} + \frac{1}{3 \cdot 5} + \frac{1}{5 \cdot 7} + \dots$
			\\\\Notice that this is equal to the series $\displaystyle\sum_{n=1}^{\infty} \frac{1}{(2n-1)(2n+1)}$. So,
			\begin{align*}
				\sum_{n=1}^{\infty} \frac{1}{(2n-1)(2n+1)} &= \sum_{n=1}^{\infty} \frac{A}{2n-1} + \frac{B}{2n+1} \\
				&\Downarrow \\
				1 &= 2An+2Bn+A-B \\
				\begin{cases}
				2An+2Bn=0 \\
				A-B=1
				\end{cases} 
				&\implies
				\begin{cases}
				A= \frac{1}{2} \\
				B= \frac{-1}{2}
				\end{cases} \\
				&= \sum_{n=1}^{\infty} \frac{1}{4n-2} - \frac{1}{4n+2} \\
				&= \left(\frac{1}{2}-\cancel{\frac{1}{6}}\right) + \left(\cancel{\frac{1}{6}}-\cancel{\frac{1}{10}}\right) + \\
				&\dots + \left(\cancel{\frac{1}{4n-5}}-\cancel{\frac{1}{4n-2}}\right) + \left(\cancel{\frac{1}{4n-2}}-\frac{1}{4n+2}\right) \\
				&=\limx{n}{\infty} \frac{1}{2} -\frac{1}{4n+2} \\
				&= \frac{1}{2} - 0 \\
				&= \frac{1}{2}
			\end{align*}
			\item $\displaystyle\sum_{n=1}^{\infty} \frac{1}{1+2+3+\dots+n}$
			\begin{align*}
				\sum_{n=1}^{\infty} \frac{1}{1+2+3+\dots+n} &= \sum_{n=1}^{\infty} \frac{1}{\sum\limits_{i=1}^{n} i} \\
				&= \sum_{n=1}^{\infty} \frac{2}{n(n+1)} \\
				&\Downarrow \\
				\frac{2}{n(n+1)} &= \frac{A}{n} + \frac{B}{n+1} \\
				2&=An+A+Bn
			\end{align*}
			\[\begin{cases}
			An+Bn=0 \\
			Bn=2
			\end{cases}=\begin{cases}
			A=2 \\
			B=-2
			\end{cases}\]
			\begin{align*}
				\sum_{n=1}^{\infty} \frac{2}{n(n+1)} &= \sum_{n=1}^{\infty} \frac{2}{n}-\frac{2}{n+1} \\
				&= \left(\frac{2}{1}-\cancel{\frac{2}{2}}\right) + \left(\cancel{\frac{2}{2}}-\cancel{\frac{2}{3}}\right) + \left(\cancel{\frac{2}{3}}-\cancel{\frac{2}{4}}\right) + \\
				&\dots + \left(\cancel{\frac{2}{n-1}}-\cancel{\frac{2}{n}}\right) + \left(\cancel{\frac{2}{n}}-\frac{2}{n+1}\right) \\
				&= \limx{n}{\infty} 2-\frac{2}{n+1} \\
				&= 2-0 \\
				&= 2
			\end{align*}
		\end{enumerate}
		\item Prove that each of the following series diverges.
		\begin{enumerate}
			\item $\displaystyle\sum_{n=1}^{\infty} \frac{n}{2n+1}$
			\begin{proof}
				Recall \textit{Theorem 3.7.1 -- The $n^{\text{th}}$-Term Test}: 
				\begin{theorem*}[\textbf{The $n$th Term Test}]
					If the series $\sum x_n$ converges, then $\lim (x_n) = 0$.
				\end{theorem*}
				Let $a_n$ be the sequence whose terms are obtained by $a_n:=\frac{n}{2n+1}$, for $n \in \N$. Then we have 
				\[\limx{n}{\infty} a_n = \limx{n}{\infty} \frac{n}{2n+1} = \limx{n}{\infty} \frac{1}{2} \text{ by L'Hospital's Rule} = \frac{1}{2} \neq 0\]
				Thus since $\limx{n}{\infty} a_n \neq 0$, we know that by \textit{Theorem 3.7.1}, $\displaystyle\sum_{n=1}^{\infty}$ is divergent.
			\end{proof}
		
			\item $\displaystyle\sum_{n=1}^{\infty} \cos \frac{1}{n^2}$
			\begin{proof}
				Let $a_n$ be the sequence whose terms are obtained by $a_n:= \cos \left(\frac{1}{n^2}\right)$, for $n \in \N$. Then we have 
				\[\limx{n}{\infty} a_n = \limx{n}{\infty} \cos \left(\frac{1}{n^2}\right) = \cos (0) = 1\]
				By \textit{The $n$th Term Test}, since $\limx{n}{\infty} a_n \neq 0$, the series $\displaystyle\sum_{n=1}^{\infty} \cos \left(\frac{1}{n^2}\right)$ is divergent.
			\end{proof}
		
			\item $\displaystyle\sum_{n=1}^{\infty} n \sin \frac{1}{n}$
			\begin{proof}
				Let $a_n$ be the sequence whose terms are obtained by $a_n:=n\sin\left(\frac{1}{n}\right)$, for $n \in \N$. Then we have
				\[\limx{n}{\infty} a_n = \limx{n}{\infty} n \sin \left(\frac{1}{n}\right)=\limx{n}{\infty} \frac{\sin\left(\frac{1}{n}\right)}{\frac{1}{n}}=\limx{n}{\infty} \frac{\frac{-\cos\left(\frac{1}{n}\right)}{n^2}}{-\frac{1}{n^2}}=\limx{n}{\infty} \cos\left(\frac{1}{n}\right) = \cos(0) = 1\]
				By using \textit{L'Hospital's Rule}. Thus by the \textit{$n$th Term Test}, since $\limx{n}{\infty} a_n \neq 0$, the sum $\displaystyle\sum_{n=1}^{\infty} n\sin\left(\frac{1}{n}\right)$ is divergent.
			\end{proof}
			\item $\displaystyle\sum_{n=1}^{\infty} \left(1-\frac{1}{n}\right)^n$
			\begin{proof}
				Let $a_n$ be the sequence whose terms are obtained by $a_n:=\left(1-\frac{1}{n}\right)^n$, for $n \in \N$. Then, we have
				\begin{align*}
					\limx{n}{\infty} a_n &= \limx{n}{\infty} \left(1-\frac{1}{n}\right)^n \\
					&= \limx{n}{\infty} e^{\ln \left(1-\frac{1}{n}\right)^n} \\
					&= \limx{n}{\infty} \exp\left\{\ln\left(1-\frac{1}{n}\right)^n\right\} \\
					&= \limx{n}{\infty} \exp \left\{n\ln\left(1-\frac{1}{n}\right)\right\} \\
					&= \limx{n}{\infty} \exp\left\{\frac{\ln\left(1-\frac{1}{n}\right)}{\frac{1}{n}}\right\} \\
					&= \limx{n}{\infty} \exp \left\{\frac{\frac{1}{1-\frac{1}{n}}\cdot \frac{1}{n^2}}{-\frac{1}{n^2}}\right\} \\
					&= \limx{n}{\infty} \exp \left\{\frac{\frac{1}{n^2-n}}{-\frac{1}{n^2}}\right\} \\
					&= \limx{n}{\infty} \exp\left\{-\frac{n^2}{n^2-n}\right\} \\
					&= \limx{n}{\infty} \exp \left\{-\frac{n}{n-1}\right\} \\
					&= \limx{n}{\infty} \exp \left\{-\frac{1}{1-\frac{1}{n}}\right\} \\
					&= \exp \left\{-\frac{1}{1-0}\right\} \\
					&= \exp (-1) \\
					&= e^{-1} \\
					&= \frac{1}{e}
				\end{align*}
				By using \textit{L'Hospital's Rule}. Thus by the \textit{$n$th Term Test}, since $\limx{n}{\infty} a_n \neq 0$, we have that the sum $\displaystyle\sum_{n=1}^{\infty} \left(1-\frac{1}{n}\right)^n$ diverges.
			\end{proof}
		\end{enumerate}
		\item 
		\begin{enumerate}
			\item Give an example of two series $\sum a_k$ and $\sum b_k$ that differ in the first five terms, yet converge to the same value.
			\\\\Consider $\displaystyle\sum_{n=1}^{\infty} \left(\frac{1}{2}\right)^n$. This series converges to 2. Also notice that $1+2+3+4-10\displaystyle\sum_{n=1}^{\infty} \left(\frac{1}{2}\right)^n = 2$. Thus the first five terms are different but converge to the same value.\\
			\item Give an example of two series $\sum a_k$ and $\sum b_k$ that differ in infinitely many terms, yet converge to the same value.
			\\\\Consider the sums
			\[\sum_{n=0}^{\infty} \frac{15}{32} \left(\frac{1}{16}\right)^n\ \text{ and }\ \sum_{n=2}^{\infty} \frac{1}{n(n+1)}\]
			Let $a_n$ and $b_n$ be the sequences whose terms are obtained by $a_n:=\frac{15}{32}\left(\frac{1}{16}\right)^n$, for $n =0,1,2,3,\dots$, and let $b_n:=\frac{1}{n(n+1)}$, for $n \in \N$. Then we have 
			\[a_n:=\left(\frac{15}{32}, \frac{15}{512}, \frac{15}{8192}, \frac{15}{131072}, \frac{15}{2097152}, \dots\right)\]
			and
			\[b_n:=\left(\frac{1}{6}, \frac{1}{12}, \frac{1}{20}, \frac{1}{30}, \frac{1}{42},\dots\right)\]
			It is clear that since the numerator of each term of $a_n$ is always $15$, and that the numerator of $b_n$ is always $1$, these two sequences are different at every term and thus differ in infinitely many terms, and thus the terms of the sums $\sum a_n$ and $\sum b_n$ also differ in infinitely many terms. Thus the first five terms of $\sum a_n$ are
			\[\frac{15}{32},\ \frac{255}{512},\ \frac{4095}{8192},\ \frac{65535}{131072},\ \frac{1048575}{2097152},\ \dots\]
			and the first five terms of $\sum b_n$ are
			\[\frac{1}{6},\ \frac{1}{4},\ \frac{3}{10},\ \frac{1}{3},\ \frac{5}{14},\ \dots\]
			However, notice that $\sum a_n$ and $\sum b_n$ converge to the  same value:
			\begin{align*}
			\sum_{n=0}^{\infty} \frac{15}{32} \left(\frac{1}{16}\right)^n &= \frac{\frac{15}{32}}{1-\frac{1}{16}} \\
			&= \frac{\frac{15}{32}}{\frac{15}{16}} \\
			&= \frac{15 \cdot 16}{15 \cdot 32} \\
			&= \frac{240}{480} \\
			&= \frac{1}{2}
			\end{align*}
			and
			\begin{align*}
			\sum_{n=2}^{\infty} \frac{1}{n(n+1)} &= \sum_{n=2}^{\infty} \frac{A}{n} + \frac{B}{n+1} \\
			1&=An+Bn+A \\
			\begin{cases}
			An+Bn=0 \\
			A=1
			\end{cases} &\implies \begin{cases}
			A=1 \\
			B=-1
			\end{cases} \\
			&\Downarrow \\
			\sum_{n=2}^{\infty} \frac{1}{n(n+1)}&= \sum_{n=2}^{\infty}\frac{1}{n}-\frac{1}{n+1} \\
			&= \left(\frac{1}{2}-\cancel{\frac{1}{3}}\right) + \left(\cancel{\frac{1}{3}}-\cancel{\frac{1}{4}}\right) + \left(\cancel{\frac{1}{4}}-\cancel{\frac{1}{5}}\right) + \\
			&\dots + \left(\cancel{\frac{1}{n-1}}-\cancel{\frac{1}{n}}\right) + \left(\cancel{\frac{1}{n}}-\frac{1}{n+1}\right) \\
			&= \limx{n}{\infty} \frac{1}{2} - \frac{1}{n+1} \\
			&= \frac{1}{2}-0 \\
			&= \frac{1}{2}
			\end{align*}
			Thus we have that
			\[\sum_{n=0}^{\infty} \frac{15}{32} \left(\frac{1}{16}\right)^n = \sum_{n=2}^{\infty} \frac{1}{n(n+1)} = \frac{1}{2}\]
			\item Give an example of two series $\sum a_k$ and $\sum b_k$ that converge to real numbers $A$ and $B$, respectively, but the series $\sum a_kb_k$ converges to a value different from $AB$.
			\\\\Consider the series $\displaystyle\sum_{n=0}^{\infty} \left(\frac{1}{2}\right)^n$ and $\displaystyle\sum_{n=0}^{\infty} \left(\frac{1}{3}\right)^n$. Then we have
			\[\sum_{n=0}^{\infty} \left(\frac{1}{2}\right)^n=\frac{1}{1-\frac{1}{2}}=\frac{2}{1}=2\]
			and
			\[\sum_{n=0}^{\infty} \left(\frac{1}{3}\right)^n = \frac{1}{1-\frac{1}{3}} = \frac{3}{2}\]
			Note that $2 \cdot \frac{3}{2}=\frac{6}{2} = 3$. But the series
			\[\sum_{n=0}^{\infty} \left(\frac{1}{2}\right)^n\left(\frac{1}{3}\right)^n = \sum_{n=0}^{\infty} \left(\frac{1}{2} \cdot \frac{1}{3}\right)^n = \sum_{n=0}^{\infty} \left(\frac{1}{6}\right)^n = \frac{1}{1-\frac{1}{6}}=\frac{6}{5}\]
			Thus we have that the product of the sums, 3, is not equal to the sum of the products, $\frac{6}{5}$.\\
			\item Give an example of a series that diverges and whose sequence of partial sums is bounded.
			\\\\Consider an alternating series, $\displaystyle\sum_{n=1}^{\infty} (-1)^n$. Then, note that $S_1:=-1,\ S_2:=-1+1=0,\ S_3:=-1+1-1=-1, S_4:=-1+1-1+1=0,\dots$. Then we have that the sequence of partial sums is bounded below by $-1$ and is bounded above by $1$. However, since this is an alternating series, we know that by the \textit{Geometric Series Test}, since $|r| = |-1|=1 \nless 1$, this series is divergent.
		\end{enumerate}
	
		\item Prove or justify, if true. Provide a counterexample, if false.
		\begin{enumerate}
			\item If $a_n$ is strictly decreasing and $\limx{n}{\infty} a_n=0$, then $\sum a_n$ converges.
			\\\\This is a false statement. Consider the series $\displaystyle\sum_{n=1}^{\infty} \frac{1}{n}$. Then, we have that $\limx{n}{\infty} \frac{1}{n} =0$, however since this is a harmonic series, we know that it is divergent, thus $\displaystyle\sum_{n=1}^{\infty} \frac{1}{n} = \infty$ is divergent.\\
			
			\item If $a_n \neq b_n$ for all $n \in \N$ and if $\sum (a_n+b_n)$ converges, then either $\sum a_n$ converges or $\sum b_n$ converges.
			\\\\This is a false statement. Consider the series $\displaystyle\sum_{n=1}^{\infty} (-1)^n$ and $\displaystyle\sum_{n=1}^{\infty} (-1)^{n+1}$. Then we have that $\displaystyle\sum_{n=1}^{\infty} (-1)^n = -1+1-1+1-1+\dots$, which is divergent by the \textit{Geometric Series Test}, and $\displaystyle\sum_{n=1}^{\infty} (-1)^{n+1} = 1-1+1-1+1-1+\dots$, which is also divergent by the \textit{Geometric Series Test}. However, $\displaystyle\sum_{n=1}^{\infty} (-1)^n+(-1)^{n+1} = 0+0+0+0+\dots = 0$ and thus converges. However, $a_n \neq b_n$ for all $n \in \N$ since
			$a_n:= -1,1,-1,1,-1,1,\dots$ and $b_n:=1,-1,1,-1,1,-1,\dots$. Thus for all $n \in \N$, either $a_n=-1 \neq 1 = b_n$, or $a_n=1\neq -1=b_n$. Thus we have that $\sum (a_n+b_n)$ converges but neither $\sum a_n$ nor $\sum b_n$ converge.\\
			
			\item Suppose $\sum (a_n+b_n)$ converges. Then $\sum a_n$ converges if and only if $\sum b_n$ converges. 
			\\\\This is a true statement since if $\sum (a_n+b_n)$ converges, then both $a_n+b_n$ converges, as was covered in our notes.\\
			
			\item If $\limx{n}{\infty} a_n=A$, then $\displaystyle\sum_{n=1}^{\infty} (a_n-a_{n+2})=a_1+a_2-2A$.
			\begin{proof}
				Notice that we can rewrite the sum $\displaystyle\sum_{n=1}^{\infty} (a_n-a_{n+2})$ as $\displaystyle\sum_{n=1}^{\infty} \left((a_n-a_{n+1})+(a_{n+1}-a_{n+2})\right)$. Now, we have that the $n$th partial sum yields a telescoping series:
				\[S_n:=[(a_1-\cancel{a_2})+(a_2-\cancel{a_3})]+ [(\cancel{a_2}-\cancel{a_3})+(\cancel{a_3}-\cancel{a_4})] + \dots + [(\cancel{a_n}-a_{n+1})+(\cancel{a_{n+1}}-a_{n+2})]\]
				So $S_n=a_1+a_2-a_{n+1}-a_{n+2}$, which yields $\limx{n}{\infty} =a_1+a_2-A-A=a_1+a_2-2A$.
			\end{proof}
			
			\item $\sum a_n$ converges if and only if $\limx{n}{\infty} a_n=0$.
			\\\\This is a false statement. Consider the series $\displaystyle\sum_{n=1}^{\infty} \frac{1}{n}$. This is the harmonic series, which we know diverges. However, $\limx{n}{\infty} \frac{1}{n} = 0$. Thus the limit is equal to 0, but the series does not converge.\\
			
			\item Changing the first few terms in a series may affect the value of the sum of the series.
			\begin{proof}
				Suppose $x_n \to x$. Then for all $\varepsilon >0$, there exists some $N \in \N$ such that if $n \geq N$, then $|x_n-x| < \varepsilon$. Now, suppose $x'_n$ is a sequence such that for $n \geq M$, then $x'_n=x_n$.
				\\\\Let $\varepsilon>0$ be given. Then there exists some $N \in \N$ such that for all $n \geq N$, $|x_n-x|<\varepsilon$. Let $N'=\max (N,M)$. Then if $n \geq N'$, we have that $|x'_n-x|<\varepsilon$. Hence $x'_n \to x$.
				\\\\Consider a convergent series $\sum x_n$. If we let $s_n=x_1+x_2+\dots +x_n$, then we have that $s_n \to s$.
				\\\\Consider the series $\sum x'_n$, where for $n \geq M$, then $x'_n=x_n$. Let $s'_n=x'_1+x'_2+\dots+x'_n$. Note that for $n \geq M$, we have $s'_n-s'_{M-1}=x'_M+\dots+x'_n=x_M+\dots + x_n$, and thus $s'_n-s'_{M-1}=s_n-s_{M-1}\to s-s_{M-1}$. Hence $s'_n \to (s-s_{M-1}+s'_{M-1})$.
			\end{proof}
			
			\item Changing the first few terms in a series may affect whether or not the series converges.
			\\\\This is a false statement. Consider the telescoping series used previously, given by $\displaystyle\sum_{n=2}^{\infty} \frac{1}{n(n+1)}$. We know this series converges to $\frac{1}{2}$. Consider changing the first few terms as follows: 
			\[\left(\frac{1}{2}-\cancel{\frac{1}{3}}\right) + \left(\cancel{\frac{1}{3}}-\cancel{\frac{1}{4}}\right) + \left(\cancel{\frac{1}{4}}-\cancel{\frac{1}{5}}\right) +\dots \to \left(1-\cancel{\frac{1}{2}}\right) + \left(\cancel{\frac{1}{2}}-\cancel{\frac{1}{4}}\right) + \left(\cancel{\frac{1}{4}}-\cancel{\frac{1}{5}}\right) + \dots\]
			Now we have that this series converges to 1, not $\frac{1}{2}$. However, despite changing the first few terms of the series, we did not change whether or not it converges. This is because the first few terms of a series can only finitely affect the sum. Thus, if a series converges, a finite change to the terms will still create a finite sum. Likewise, if the series diverges, a finite change will not allow the series to converge to a finite sum.\\
			
			\item If $\sum a_n$ converges and $\limx{n}{\infty} \frac{a_n}{b_n}=0$, then $\sum b_n$ converges.
			\\\\This is a false statement. Consider the series $\displaystyle\sum_{n=1}^{\infty} \left(\frac{1}{2}\right)^n$ and $\displaystyle\sum_{n=1}^{\infty} \frac{1}{n}$, which we note is the harmonic series. Then we have that $\displaystyle\sum_{n=1}^{\infty} \left(\frac{1}{2}\right)^n = \frac{1}{1-\frac{1}{2}} = 2$ since it is a geometric series. Thus $\sum a_n$ converges. Also, notice that
			\[\limx{n}{\infty} \left(\frac{\left(\frac{1}{2}\right)^n}{\frac{1}{n}}\right)=0\]
			However, since $\sum b_n = \displaystyle\sum_{n=1}^{\infty} \frac{1}{n}$, which is the harmonic series, we know that it is divergent, and thus if $\sum a_n$ converges and $\limx{n}{\infty} \frac{a_n}{b_n}=0$, $\sum b_n$ can still be divergent.
		\end{enumerate}
	\end{enumerate}
\end{document}
